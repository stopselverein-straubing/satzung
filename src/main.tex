\documentclass[a4paper,12pt]{scrartcl}

\usepackage[utf8]{inputenc}
\usepackage[T1]{fontenc}
\usepackage[ngerman]{babel}
\usepackage[official]{eurosym}
\usepackage{hyperref}

\renewcommand{\autodot}{}

\newcommand{\vereinsName}{Stopselverein Straubing 2021}
\newcounter{paragraphNumber}

\newcommand{\svsrParagraph}[1]{%
    \stepcounter{paragraphNumber}
    \paragraph{(\Roman{paragraphNumber}) #1}%
}


\title{Vereinssatzung}
\author{\vereinsName}
\date{\today}

\begin{document}

    \maketitle

    \svsrParagraph{Vorstandschaft}
    Die Vorstandschaft des \emph{\vereinsName} setzt sich zusammen aus:
    \begin{itemize}
        \item Erster Vorstand
        \item Zweiter Vorstand
        \item Kassier
    \end{itemize}

    \svsrParagraph{Wahl Vorstandschaft}
    Die Vorstandschaft ist alle \textbf{vier Jahre}
    bei der Jahreshauptversammlung zu wählen.
    Die Wahl kann sowohl geheim als auch mit Zustimmung der anwesenden Mitglieder
    öffentlich stattfinden.
    Es wird dabei der erste Vorstand gewählt.
    Derjenige, welcher die meisten Stimmen erhält, gewinnt die Wahl.
    Der erste Vorstand beruft anschließend die übrige Vorstandschaft.

    \svsrParagraph{Jahreshauptversammlung}
    Jedes Jahr wird eine Jahreshauptversammlung mit allen Mitglieder abgehalten.
    Dabei müssen zu Beginn alle aktuell ausstehenden Strafen bezahlt werden.
    Anschließend wird das Kassenbuch geprüft.

    \svsrParagraph{Gründungsmitglieder}
    Alle bei der Gründungsversammlung anwesenden Personen werden als Gründungsmitglieder aufgenommen.
    Die dabei anwesenden Mitglieder werden vom Schriftführer der Gründungsveranstaltung namentlich im Protokoll festgehalten.
    Zudem wird mit der Mehrheit der anwesenden Personen der erste Vorstand gewählt.

    \svsrParagraph{Neuaufnahme Mitglieder}
    Ein Mitglied, welches aufgenommen werden will, muss einen entsprechenden Mitgliedsantrag ausfüllen.
    Der erste Vorstand entscheidet anschließend darüber,
    ob diese Person als Mitglied aufgenommen wird oder nicht.

    \svsrParagraph{Vermögensverwaltung}
    Besteht ein Bankkonto für den \emph{\vereinsName}, kann Geld nur abgehoben beziehungsweise überwiesen werden,
    wenn der erste Vorstand oder der Kassier ihre Zustimmung durch ihre Unterschrift gegeben haben.
    Der Kassier muss die Kassenbelege zu jeder Kassenprüfung mit vorlegen.
    Die Vorstandschaft kann ohne Zustimmung der übrigen Mitglieder nur bis zu einem Betrag von \euro{1000,00} verfügen.

    \svsrParagraph{Vermögensverteilung bei Auflösung}
    Wird der \emph{\vereinsName} aufgelöst, so fällt das Vermögen zu gleichen Teilen an die Mitglieder.
    Sind bei der Auflösung Schulden vorhanden, so haftet die \textbf{Vorstandschaft}.

    \svsrParagraph{Austritt}
    Tritt ein Mitglied aus dem \emph{\vereinsName} aus oder es wird aus diesem ausgeschlossen,
    so hat dieses Mitglied keinerlei Anspruch auf Geld oder sonstige Gegenstände.

    \svsrParagraph{Vereinseigentum}
    Die Vereinsfahne sowie das Stammtischzeichen sind Eigentum des \emph{\vereinsName}.
    Bei Verlust haftet der gesamte Club.
    Bei Beschädigung haftet lediglich der Schuldige.
    Dies trifft auch für alle anderen Gegenstände zu, welche Eigentum des \emph{\vereinsName} sind.

    \svsrParagraph{Mitgliedsbeitrag}
    Jedes Mitglied muss jährlich einen Mitgliedsbeitrag an den \emph{\vereinsName} entrichten.
    Die Höhe ist im \autoref{sec:fees} in \autoref{tab:fees} festgelegt.
    Der Mitgliedsbeitrag kann bei jeder Jahreshauptversammlung erhöht beziehungsweise gesenkt werden,
    wenn mindesten 50\% der anwesenden Mitglieder zustimmen.
    Wer den Mitgliedsbeitrag bis zum Ende des Beitragsjahres nicht bezahlt hat,
    kann mit der Zustimmung der Vorstandschaft vom Verein ausgeschlossen werden.

    \svsrParagraph{Mitführpflicht des Stopsels}
    Jedes Mitglied erhält einen Stopsel, welcher gekennzeichnet ist.
    Jedes Mitglied ist berechtigt, den Stopsel eines anderen Mitglieds zu fordern.

    \svsrParagraph{Ausnahmen von der Mitführpflicht}
    Ein Mitglied muss den Stopsel in folgenden Fällen nicht mitführen:
    \begin{enumerate}
        \item Im Dress beim Fußballspiel (Spieler/Schiedsrichter/Trainer)
        \item Im Badeoutfit beim Baden
        \item In Notfällen bei Feuerwehr, Rettungsdienst oder THW
        \item Bei Krankheit im Krankenhaus
        \item In Outfits, welche keine Taschen haben
    \end{enumerate}

    \svsrParagraph{Strafen}
    Kann ein Mitglied den Stopsel nicht vorzeigen oder weigert es sich,
    den Stopsel vorzuzeigen, so ist eine Strafgebühr zu entrichten.
    Ebenso muss eine Strafgebühr entrichtet werden, wenn der Stopsel schwer beschädigt wird
    oder bei Verlust des Stopsels.
    Die Höhe der Strafgebühren sind im \autoref{sec:fees} in \autoref{tab:penalties} festgelegt.
    Bei Verlust des Stopsels kann gegen eine Gebühr ein neuer Stopsel beschafft werden.
    Die Gebühr ist im \autoref{sec:fees} in \autoref{tab:fees} festgelegt.

    \svsrParagraph{Meldung von Strafen}
    Alle Strafen müssen der Vorstandschaft umgehend gemeldet werden.
    Wird ein Vergehen innerhalb von 24 Stunden der Vorstandschaft nicht gemeldet,
    so verfällt dieses.

    \svsrParagraph{Zahlung von Strafen}
    Die Zahlung der Strafen muss umgehend an den Kassier des \emph{\vereinsName} erfolgen.
    Dabei wird jedem Mitglied eine Frist von \textbf{drei Monaten} gestattet.
    Wird eine Strafgebühr nicht rechtzeitig bezahlt,
    so kann bei der nächsten Jahreshauptversammlung über den Ausschluss dieses Mitglieds abgestimmt werden.

    \svsrParagraph{Änderung der Satzung}
    Die Satzung kann in jeder Jahreshauptversammlung geändert werden.
    Dabei müssen mindestens zwei Drittel der anwesenden Mitglieder der Änderung zustimmen.
    Dieser Absatz kann dabei nicht geändert werden.

    \clearpage
    \appendix

    \section{Gebührenübersicht}
\label{sec:fees}

\newcommand{\svsrEntry}[2]{%
    #1 & \emph{#2} \\
    \hline%
}

\begin{table}[htb]
    \centering
    \begin{tabular}{|l|c|}
        \hline
        \textbf{Beschreibung} & \textbf{Gebühr in \euro} \\
        \hline
        \svsrEntry{Jährlicher Mitgliedsbeitrag}{20}
        \svsrEntry{Neubeschaffung eines Stopsels}{5}
    \end{tabular}
    \caption{Festgelegte Gebühren}
    \label{tab:fees}
\end{table}

\begin{table}[htb]
    \centering
    \begin{tabular}{|l|c|}
        \hline
        \textbf{Vergehen} & \textbf{Strafe in \euro} \\
        \hline
        \svsrEntry{Verletzung der Mitführpflicht}{2}
        \svsrEntry{Schwere Beschädigung des Stopsels}{10}
        \svsrEntry{Verlust des Stopsels}{20}
    \end{tabular}
    \caption{Strafenkatalog}
    \label{tab:penalties}
\end{table}


\end{document}
